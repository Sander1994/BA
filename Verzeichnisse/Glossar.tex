\newglossaryentry{Kästchen}
{
	name=Kästchen,
	plural=Kästchen,
	description={Ein Kästchen beschreibt ein Rechteck des Spielbretts von \textit{"Ganz schön clever"}, welches ausgefüllt werden kann, wenn bestimmte Bedingungen erfüllt sind.}
}
\newglossaryentry{Feld}
{
	name=Feld,
	plural=Felder,
	description={Ein Feld ist eines der farbigen Felder (gelb, blau, grün, orange, lila) auf dem Spielbrett von \textit{"Ganz schön clever"}.}
}
\newglossaryentry{Extra Wahl}
{
	name=Extra Wahl,
	plural=Extra Wahlen,
	description={Eine Extra Wahl beschreibt eine Wahl mithilfe der Extra Wahl Boni oder vom Silbertablett eines Mitspielers.}
}
\newglossaryentry{Wahl}
{
	name=Wahl,
	plural=Wahlen,
	description={Eine Wahl beschreibt den Vorgang bei dem ein Würfel oder eine Boni gewählt wird, um eines der Kästchen auf dem Spielbrett auszufüllen.\newpage}
}
\newglossaryentry{normale Wahl}
{
	name=normale Wahl,
	plural=normale Wahlen,
	description={Eine normale Wahl beschreibt die Wahl eines Würfels nach einem eigenen Wurf oder eine Wahl mithilfe eines der Boni, welcher es erlauben direkt eines der Kästchen in einem der farbigen Felder auszufüllen.}
}
\newglossaryentry{Bonusrunde}
{
	name=Bonusrunde,
	plural=Bonusrunden,
	description={Eine Bonusrunde ist eine Runde bei der der Rundenablauf nicht wie üblich inkrementiert wird. Diese Einteilung ist notwendig, da bei Wahlen mit Boni andere Regeln gelten, als bei Wahlen nach dem eigenen Wurf oder nach Wahlen vom Silbertablett eines Mitspielers.}
}
\newglossaryentry{ungültige Aktion}
{
	name=ungültige Aktion,
	plural=ungültige Aktionen,
	description={Eine ungültige Aktion ist eine Aktion, die im normalen Spielablauf nicht stattfinden würde. Ungültige Aktionen treten auf, wenn dem Modell keine gültige Aktion zur Auswahl steht.}
}
\newglossaryentry{ungültiger Würfel}
{
	name=ungültiger Würfel,
	plural=ungültige Würfel,
	description={Ein ungültiger Würfel ist ein Würfel, der nicht mehr zur Wahl steht und nicht mehr geworfen wird.}
}
\newglossaryentry{Modell}
{
	name=Modell,
	plural=Modelle,
	description={Ein Modell beschreibt die Entität der Künstlichen Intelligenz, welche im Stande ist, Vorhersagen zu treffen}
}
\newglossaryentry{Agent}
{
	name=Agent,
	plural=Agenten,
	description={Ein Agent ist die ausführende Instanz der Künstlichen Intelligenz. Er nimmt die Umgebung wahr und führt Aktionen in dieser aus.}
}