\centering
\includegraphics[width=.8\textwidth]{framework/Logo_HS_Coburg}

\begin{Large}
  Hochschule für angewandte Wissenschaften Coburg\\
  Fakultät Elektrotechnik und Informatik\par
\end{Large}
\vspace{1.5cm}

\Large{Studiengang: \Studiengang}
\vspace{1.5cm}

\Large{\DocumentType}
\vspace{1cm}

\Huge{\Titel}
\vspace{2cm}

\huge{\Autorenname}
\vspace{2cm}

\Large{Abgabe der Arbeit: \Abgabe}

\Large{Betreut durch:}

\Large{\Betreuer, Hochschule Coburg}

\newpage
\begin{abstract}
\raggedright
Zunehmend prägen das maschinelle Lernen und KI die Arbeit und das Leben der Menschen. Besonders präsent sind im Jahr 2023 unter Anderem potente Chatbots wie ChatGPT 4. Solche Tools ermöglichen es Benutzern komplexe sowie komplizierte Aufgaben deutlich einfacher und schneller abzuarbeiten. Hervorzuheben ist hierbei auch, dass man mit solchen Tools deutlich weniger Fachwissen benötigt, um in einem Bereich Aufgaben effizient lösen zu können, da es einem eine Vielzahl von Informationen zum gewünschten Thema auf Anfrage bereitstellen kann. Je komplexer das Problem oder die Fragestellung allerdings sind, desto unverlässlicher werden diese Tools. Man muss seine Anfragen deshalb möglichst präzise formulieren und die Problemstellung in für das Tool angemessene Teilaufgaben zerlegen. In anderen Bereichen verhält es sich ähnlich.

Auch in der Spielentwicklung spielen maschinelles Lernen und KI schon seit langem eine bedeutende Rolle. In den meisten Spielen gibt es sogenannte Bots, welche man als KI bezeichnen kann, sobald sie in der Lage sind sich eigenständig an ihre Umgebung anzupassen. Diese Bots sollen bestimmte Aufgaben im Spiel erfüllen um den Spieler zu unterstützen oder im als Widersacher zu dienen. Je komplexer die Aufgabe, umso schwerer ist es einen solchen Bot zu erstellen, welcher die Aufgabe auf zufriedenstellende Weise erfüllen kann.

Das Gesellschaftsspiel "Ganz schön clever" ist ein Würfelspiel, welches eine hohe Komplexität aufweist. Diese kommt vor allem durch die vielen Aktionsmöglichkeiten des Spielers und die multiplen zusammenhängen innerhalb des Belohnungssystems zustande. Außerdem weist es eine hohe Stochastizität auf, welche die Komplexität weiter erhöht.
Ziel dieser Arbeit ist es eine KI beziehungsweise einen Bot für dieses Spiel zu entwickeln, der das Spiel effizient spielen kann, sowie zu analysieren welche Aspekte der Entwicklung dabei relevant und zu beachten sind.

Dazu mussten Spielumgebung sowie KI zunächst implementiert werden. Dies geschah mithilfe von Bibliotheken wie Stable-Baselines3 und Gymnasium.
Insgesamt ergab sich dabei, dass sich mithilfe des PPO-Algorithmus von Stable-Balseslines3 auf relativ einfache Weise ein effizientes Modell für das Spiel entwickeln lässt.
\end{abstract}

\newpage
\begin{abstract}
\raggedright
\renewcommand{\abstractname}{Abstract}
Increasingly, machine learning and AI are shaping people's work and lives. Particularly prominent in 2023 are potent chatbots like ChatGPT 4. Such tools enable users to handle complex and complicated tasks much more easily and quickly. It is also noteworthy that with such tools, significantly less expertise is required to efficiently solve tasks in a field, as they can provide a wealth of information on the desired topic upon request. However, the more complex the problem or question, the less reliable these tools become. Therefore, one must formulate their inquiries as precisely as possible and break down the problem into subtasks appropriate for the tool. The situation is similar in other areas.

In game development, machine learning and AI have long played a significant role. In most games, there are so-called bots, which can be considered AI, as soon as they are able to adapt independently to their environment. These bots are intended to fulfill certain tasks in the game to support the player or serve as an adversary. The more complex the task, the more difficult it is to create a bot that can satisfactorily accomplish the task.

The board game "Ganz schön clever" is a dice game that exhibits high complexity. This complexity arises primarily from the player's many action options and the multiple interconnections within the reward system. Additionally, it has a high level of stochasticity, which further increases the complexity.
The goal of this work is to develop an AI or a bot for this game that can play the game efficiently, as well as to analyze which aspects of development are relevant and need to be considered.

To do this, both the game environment and the AI had to be implemented first. This was done using libraries like Stable-Baselines3 and Gymnasium.
Overall, it was found that an efficient model for the game can be relatively easily developed using the PPO algorithm from Stable-Baselines3.
\end{abstract}