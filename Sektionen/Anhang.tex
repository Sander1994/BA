\section{Code für die Attribute der Spielumgebung}
\begin{minipage}{\linewidth}
Code 18 zeigt die Klassenattribute, die für das bilden einer Punktestandhistorie relevant sind:
\vspace{0.5cm}
\begin{lstlisting}[caption={Klassenattribute für die Nachvollziehbarkeit von Punkteständen}, basicstyle=\ttfamily]
self.score
self.score_history
self.initialized
\end{lstlisting}
\end{minipage}

Das Attribut \texttt{score} repräsentiert den aktuellen Punktestand der Spielumgebung. Es wird verwendet, um erreichte Punktestände in die \texttt{score\_history} einzutragen. Das Attribut \texttt{score\_history} ist eine Historie über die erreichten Punktestände in den einzelnen Episoden beziehungsweise Spieldurchläufen. Das Attribut \texttt{initialized} wird verwendet, um zu gewährleisten, dass nur Einträge in der Punktestandhistorie eingetragen werden, nachdem ein Spiel abgeschlossen wurde. Es trifft eine Aussage darüber ob die Spielumgebung bereits einmal initialisiert wurde oder nicht.\\

\begin{minipage}{\linewidth}
Code 19 zeigt die Klassenattribute der Flags für die Belohnungen der farbigen Felder. Sind diese gesetzt und sind die entsprechenden Kästchen für die Belohnung ausgefüllt, wurde die Belohnung bereits ausgeschüttet und wird es nicht erneut, bis die Attribute nach dem Spiel zurückgesetzt werden:
\vspace{0.5cm}
\begin{lstlisting}[caption={Klassenattribute für Belohnungsflags}, basicstyle=\ttfamily]
self.yellow_reward_flags = {"row": [False] * 4, "col": ...}
self.blue_reward_flags = {"row": [False] * 3, "col": ...}
self.blue_count_reward_flags = [False] * 12
self.green_reward_flags = [False] * 11
self.orange_reward_flags = [False] * 11
self.purple_reward_flags = [False] * 11
\end{lstlisting}
\end{minipage}

\begin{minipage}{\linewidth}
Code 20 zeigt die Klassenattribute für Punktestände der einzelnen farbigen Felder. Die Punktewerte werden aufaddiert, sobald Punkte im entsprechenden Feld erspielt worden sind und am Ende genutzt, um den Wert der Fuchs-Boni zu bestimmen [siehe Unterabschnitt 2.1.1]:
\vspace{0.5cm}
\begin{lstlisting}[caption={Klassenattribute für erreichte Punktestände der einzelnen farbigen Felder}, basicstyle=\ttfamily]
self.yellow_field_score
self.blue_field_score
self.green_field_score
self.orange_field_score
self.purple_field_score
\end{lstlisting}
\end{minipage}

\begin{minipage}{\linewidth}
Code 21 zeigt die Klassenattribute für die Anzahl an gewählten Kästchen in den verschiedenen farbigen Feldern. Wenn ein Kästchen in einem der Felder ausgefüllt wird, wird der Wert des entsprechenden Attributes inkrementiert. Diese Attribute dienen nicht dem Spielablauf selbst, sondern der Nachvollziehbarkeit der Strategie des Modells:
\vspace{0.5cm}
\begin{lstlisting}[caption={Klassenattribute für die Anzahl an gewählte Kästchen innerhalb der farbigen Feldern}, basicstyle=\ttfamily]
self.picked_yellow
self.picked_blue
self.picked_green
self.picked_orange
self.picked_purple
\end{lstlisting}
\end{minipage}

\newpage
\section{Pseudocode für Methoden der Implementierung}
\begin{minipage}{\linewidth}
Code 22 zeigt die Funktionsweise der Funktion zur Visualisierung der Spielumgebung (\texttt{render method}) mithilfe von Pseudocode. Die Visualisierung dient einer verbesserten Nachvollziehbarkeit der Ereignisse innerhalb der Spielumgebung:
\vspace{0.5cm}
\begin{lstlisting}[caption={Methode zur Visualisierung der Spielumgebung}]
render():
	Zeige alle relevanten Attribute und Merkmale der Umgebung an
\end{lstlisting}
\end{minipage}

\begin{minipage}{\linewidth}
Code 23 zeigt die Funktionsweise der Methode zur Generierung des Beobachtungsraumes (\texttt{\_get\_obs method}) mithilfe von Pseudocode. Diese Methode vereinfacht es, den Beobachtungsraum zu generieren, welcher jedes Mal benötigt wird, wenn ein Schritt in der Umgebung ausgeführt wird:
\vspace{0.5cm}
\begin{lstlisting}[caption={Methode zur Generierung des Beobachtungsraumes}]
_get_obs():
	Erstelle Numpy Arrays für farbige Felder
	Erstelle Numpy Array für Würfelergebnisse
	Erstelle Numpy Array für ungültige Würfel
	Füge Arraywerte für Boni und Rundenzahl hinzu
	Beobachtungsraum = Verbinde alle Arrays miteinander

	return Beobachtungsraum
\end{lstlisting}
\end{minipage}

\begin{minipage}{\linewidth}
Code 24 zeigt die Funktionsweise der Methode zur Initialisierung der Spielumgebungen (\texttt{\_init\_envs method}) mithilfe von Pseudocode. Mit dieser Methode werden Spielumgebungen initialisiert und einer Vektorumgebung zugewiesen. Diese Vektorumgebung ermöglicht es mehrere Spielumgebungen gleichzeitig zu betreiben. Außerdem werden Variablen für die Nachvollziehbarkeit der Abläufe innerhalb der Spielumgebung (sogenannte Historien) initialisiert:
\vspace{0.5cm}
\begin{lstlisting}[caption={Methode zur Initialisierung der Spielumgebungen},morekeywords={Anzahl, Punktestände, Fehlversuche}]
_init_envs(Anzahl, Punktestände, Fehlversuche):
	_init():
		Spielumgebung = Initialisieren eine Spielumgebung
		Setze Aktionsmasker für die Spielumgebung
		return Spielumgebung
	Initialisiere Verktorumgebung(Anzahl, _init)
		if not Punktestände and not Fehlversuche:
			return Vektorumgebung
		if Punktestände and not Fehlversuche:
			Erstelle Variablen für Punkteständehistorie
		if not Punktestände and Fehlversuche:
			Erstelle Variablen für Fehlversuchehistorie
		if Punktestände and Fehlversuche:
			Erstelle Variablen für Punkteständehistorie
			Erstelle Variablen für Fehlversuchehistorie
	return Vektorumgebung, Variablen für Historien
\end{lstlisting}
\end{minipage}

\begin{minipage}{\linewidth}
Code 25 zeigt die Funktionsweise der Methode zum Anwenden der Aktionsmaske (\texttt{mask\_fn method}) mithilfe von Pseudocode. Die Methode dient dazu dem Modell die Aktionsmasken der einzelnen Spielumgebungen zu übergeben:
\vspace{0.5cm}
\begin{lstlisting}[caption={Methode zum Anwenden der Aktionsmaske},morekeywords={Spielumgebung}]
mask_fn(Spielumgebung):
	Caste Spielumgebung in benutzerdefinierte Spielumgebung
	return Aktionsmaske der Spielumgebung
\end{lstlisting}
\end{minipage}

\begin{minipage}{\linewidth}
Code 26 zeigt die Funktionsweise der Methoden zum Erstellen von Einträgen und Historien (\texttt{make\_fail\_entries method, make\_score\_entries method, make\_fail\_history\_entry method, make\_score\_history\_entry method}) mithilfe von Pseudocode. Diese Methoden erstellen Einträge und Historien von erzielten Punkteständen und der Anzahl getätigter ungültiger Aktionen innerhalb abgeschlossener Spiele:
\vspace{0.5cm}
\begin{lstlisting}[caption={Methoden zum Erstellen von Einträgen für Historien},morekeywords={Punktebelohnungen, Anzahl, Fehlversuche}]
make_fail_entries(Punktebelohnungen, Anzahl, Fehlversuche):
	if Punktebelohnung < 0:
		Inkrementiere Fehlversuche der Umgebung

make_score_entries(Punktebelohnungen, Anzahl, Punktestände):
	if Punktebelohnung > 0:
		Addiere Punktebelohnung zum Punktestand der Umgebung

make_fail_history_entry(Fehlversuche, Fehlversuchshistorie)
	if Umgebung terminiert:
		Hänge Fehlversuche an Fehlversuchshistorie an

make_score_history_entry(Punktestände, Punktehistorie)
	if Umgebung terminiert:
		Hänge Punktestand der Umgebung Punktehistorie an
\end{lstlisting}
\end{minipage}

\begin{minipage}{\linewidth}
Code 27 zeigt die Funktionsweise der Methode zum Plotten von Historien (\texttt{plot\_history method}) mithilfe von Pseudocode. Diese Methode visualisiert die generierten Historien:
\vspace{0.5cm}
\begin{lstlisting}[caption={Methode zum Plotten von Historien},morekeywords={Historien}]
plot_history(Historie):
	Plotte jeden Eintrag der Historie
	Setze Titel
	Setze Labels
	Zeige Grafik an
\end{lstlisting}
\end{minipage}
\section{Ausführlicher Code der wesentlichen Methoden der Implementierung}
Code 28 zeigt die Funktionsweise der Schritt-Methode (\texttt{step method}) der Spielumgebung mithilfe von Pseudocode. Diese Methode führt Spielschritte beziehungsweise Aktionen in der Spielumgebung aus:
\vspace{0.5cm}
\begin{lstlisting}[caption={Schritt-Methode},morekeywords={Aktion}]
step(Aktion):
	Setze Methodenattribute zurück
	
	if not Extra-Wahl:
		if Aktion im gelben Feld:
			Fülle entsprechendes gelbes Kästchen aus
			if Gelbes-Kreuz-Boni >= 1
				Bonusrunde = True
				Gelbes-Kreuz-Boni -= 1
			if not Bonusrunde:
				Setze entsprechende Würfel auf ungültig
		if Aktion im blauen Feld:
			Fülle entsprechendes blaues Kästchen aus
			if Blaues-Kreuz-Boni >= 1:
				Bonusrunde = True
				Blaues-Kreuz-Boni -= 1
			if not Bonusrunde:
				Setze entsprechende Würfel auf ungültig
		if Aktion im grünen Feld:
			Fülle entsprechendes grünes Kästchen aus
			if Grünes-Kreuz-Boni >= 1:
				Bonusrunde = True
				Grünes-Kreuz-Boni -= 1
			if not Bonusrunde:
				Setze entsprechende Würfel auf ungültig
		if Aktion im orangene Feld:
			Fülle entsprechendes orangenes Kästchen aus
			if Orangene-Feld-Boni >= 1:
				Bonusrunde = True
				Orangene-Feld-Boni -= 1
			if not Bonusrunde:
				Setze entsprechende Würfel auf ungültig
		if Aktion im lila Feld:
			Fülle entsprechendes lila Kästchen aus
			if Lila-Sechs-Boni >= 1:
				Bonusrunde = True
				Lila-Sechs-Boni -= 1
			if not Bonusrunde:
				Setze entsprechende Würfel auf ungültig
	
	if Extra-Wahl:
		Bonusrunde = True
		if Extra-Wahl-Bonus benutzt:
			Extra-Wahl-Boni -= 1
		Finde Kästchen zum ausfüllen und fülle es aus
		Setze gewählten Würfel auf ungültig
		if Extra-Wahl-Boni <= 0 or Wahl erfolgte vom Silbertablett:
			Runde wird inkrementiert
	
	if Neu-Würfeln-Bonus wird benutzt:
		Werfe Würfel neu
		Neu-Würfeln-Boni -= 1
		Aktualisiere Aktionsmaske
		return Beobachtungsraum, Punktebelohnung, Spiel terminiert?
	
	if Statt Extra-Wahl-Boni gepasst wird:
		Runde wird inkrementiert
		Prüfe ob Spiel abgeschlossen wurde
		return Beobachtungsraum, Punktebelohnung, Spiel terminiert?
	if ungültige Aktion gewählt da keine gültigen Aktionen vorhanden:
		Nichts tun
	
	if not Bonusrunde:
		Runde wird inkrementiert
		Prüfe ob Spiel abgeschlossen wurde
	Punktebelohnung += Fuchsbonipunktebelohnung
	Punktebelohnung += erspielte Belohnung in diesem Schritt
	Aktualisiere Aktionsmaske
	return Beobachtungsraum, Punktebelohnung, Spiel terminiert?		
\end{lstlisting}

\begin{minipage}{\linewidth}
Code 29 zeigt die Funktionsweise der Methode zur Überprüfung der freigespielten Belohnungen (\texttt{check\_rewards method}) mithilfe von Pseudocode. Diese Methode überprüft jedes Mal, nachdem ein Kästchen in der Spielumgebung ausgefüllt worden ist, ob und welche Belohnung freigespielt wurde und fügt diese dem Inventar des Spielers hinzu:
\vspace{0.5cm}
\begin{lstlisting}[caption={Methode zur Überprüfung der freigespielten Belohnungen}]
check_rewards():
	Überprüfe ob im gelben Feld Belohnungen freigeschaltet wurden
	Schalte im gelben Feld freigeschaltete Belohnungen frei
	
	Überprüfe ob im blauen Feld Belohnungen freigeschaltet wurden
	Schalte im blauen Feld freigeschaltete Belohnungen frei
	
	Überprüfe ob im grünen Feld Belohnungen freigeschaltet wurden
	Schalte im grünen Feld freigeschaltete Belohnungen frei
	
	Überprüfe ob im orangenen Feld Belohnungen freigeschaltet wurden
	Schalte im orangenen Feld freigeschaltete Belohnungen frei
	
	Überprüfe ob im lila Feld Belohnungen freigeschaltet wurden
	Schalte im lila Feld freigeschaltete Belohnungen frei

return Punktebelohnung
\end{lstlisting}
\end{minipage}

Code 30 zeigt Die Funktionsweise der Methode zur Generierung der Aktionsmaske (\texttt{valid\_action\_mask method}) mithilfe von Pseudocode. Einsen stehen für gültige Aktionen und Nullen für ungültige. Die Wahlwahrscheinlichkeit des Modells für ungültige Aktionen wird auf null gesetzt:
\vspace{0.5cm}
\begin{lstlisting}[caption={Methode zur Generierung der Aktionsmaske}]
valid_action_mask():
	Setze alle Werte auf Eins
	
	Werte für bereits ausgefüllte gelbe Kästchen = 0
	Werte für gelbe Kästchen ohne passende Würfelergebnisse = 0
	
	Werte für bereits ausgefüllte blaue Kästchen = 0
	Werte für blaue Kästchen ohne passende Würfelergebnisse = 0
	
	Werte für bereits ausgefüllte grüne Kästchen = 0
	Werte für grüne Kästchen ohne passende Würfelergebnisse = 0
	
	Werte für bereits ausgefüllte orangene Kästchen = 0
	Werte für orangene Kästchen ohne passende Würfelergebnisse = 0
	
	Werte für bereits ausgefüllte lila Kästchen = 0
	Werte für lila Kästchen ohne passende Würfelergebnisse = 0
	
	Werte für Akionen mithilfe ungültiger Würfel = 0
	
	if not Extra-Wahl or ungültige Extra-Wahl:
		Werte von 122 bis 243 & Wert für das Aussetzen = 0
	if gültige Extra-Wahl:
		Werte von 0 bis 121 = 0
		Wert für das Aussetzen = 1
	
	if Neu-Würfeln-Boni <= 0 or Extra-Wahl:
		Wert für Neu-Würfeln-Bonus = 0
	
	if einer der Boni zum direkten Ankreuzen von Kästchen >= 1:
		Alle Werte = 0
	
	if Gelbes-Kreuz >= 1:
		Werte von 0 bis 15 = 1
	if Blaues-Kreuz >= 1:
		Werte von 16 bis 27 = 1
	if Grünes-Kreuz >= 1:
		Werte von 28 bis 38 = 1
	if (Orangene-Vier or Orangene-Fünf or Orangene-Sechs) >= 1:
		Werte von 39 bis 49 = 1
	if Lila-Sechs >= 1:
		Werte von 50 bis 60 = 1
	
	if Alle Werte außer 246 == 0:
		Wert für ungültige Aktion = 1
	else:
		Wert für ungültige Aktion = 0
	
	return Aktionsmaske
\end{lstlisting}

Code 31 zeigt die Funktionsweise der Methode zum Inkrementieren des Runden-Systems (\texttt{increment\_rounds method}) mithilfe von Pseudocode. Diese Methode ist dafür zuständig das Runden-System des Spiels voranzutreiben:
\vspace{0.5cm}
\begin{lstlisting}[caption={Methode zum Inkrementieren des Runden-Systems}]
increment_rounds():
	if Extra-Wahl nach eignener Runde:
		Wahl vom Silbertablett = True
		Extra Wahl nach eigener Runde = False
		Setze alle Würfel auf gültig
		Würfele Würfel neu
		Setze drei zufällige Würfel auf ungültig
	elif Wahl vom Silbertablett:
		if Extra-Wahl-Boni >= 1:
			Extra-Wahl nach Wahl vom Silbertablett = True
		else:
			Schalte Bonus für erreichte Runde frei
			Werfe Würfel neu
			Prüfe ob Spiel abgeschlossen wurde
		Wahl vom Silbertablett = False
		Setze alle Würfel auf gültig
	elif Extra-Wahl nach Wahl vom Siblertablett:
		Extra-Wahl nach Wahl vom Silbertablett = False
		Werfe Würfel neu
		Setze alle Würfel auf gültig
		Schalte Bonus für erreichte Runde frei\\
		Prüfe ob Spiel abgeschlossen wurde
	elif Wurf in Runde >= 3:
		Rundenanzahl -= 1
		Wurf in Runde = 1
		Setze alle Würfel auf gültig
		if Extra-Wahl-Boni >= 1:
			Extra-Wahl nach eigener Runde = True
		else:
			Wahl vom Silbertablett = True
			Werfe Würfel neu
			Setze drei zufällige Würfel auf ungültig	
	else:
		Wurf in Runde += 1
		Werfe Würfel neu
\end{lstlisting}

\begin{minipage}{\linewidth}
	Code 32 zeigt die Funktionsweise der Methode zum Anlernen des Modells (\texttt{model\_learn method}) mithilfe von Pseudocode. Es werden Hyperparameter festgelegt, welche genutzt werden, um ein MaskablePPO-Modell zu generieren und mithilfe der Spielumgebung zu trainieren. Alternativ kann ein bereits trainiertes Modell geladen und weiter trainiert werden:
	\vspace{0.5cm}
	\begin{lstlisting}[caption={Methode zu Anlernen des Modells},morekeywords={Hyperparameter}]
		model_learn(Hyperparameter):
		Initialisiere Spielumgebungen
		Modell = Initialisiere MaskablePPO-Modell mit Hyperparamter
		if Modellname in Hyperparameter:
		Modell = Lade Modell mit Modellname
		Modell.Spielumgebungen = Initialisieren Spielumgebungen
		Setze Entropie-Koeffizient
		Lerne Modell an
		Setze Gamma für Vorhersagen
		Setze Entropie-Koeffizient für Vorhersagen
		Lerne Modell erneut an
		Speichere Modell
	\end{lstlisting}
\end{minipage}

\begin{minipage}{\linewidth}
Code 33 zeigt die Funktionsweise der Methode zum Vorhersagen von Aktionen mithilfe des Modells (\texttt{model\_predict method}) unter Verwendung von Pseudocode. Diese Methode ermöglicht es, mithilfe des Modells Vorhersagen über günstige Aktionen in einem gegebenen Zustand der Spielumgebung vorherzusagen und anhand dessen einen Spielverlauf zu simulieren:
\vspace{0.5cm}
\begin{lstlisting}[caption={Methode zum Vorhersagen von Aktionen mithilfe des Modells},morekeywords={Schrittanzahl}]
model_predict(Schrittanzahl):
	Lade Modell
	Initialisiere Spielumgebungen und Historien
	Setze Beobachtungsraum
	for i in range(Schrittanzahl):
		Aktualisiere Aktionsmaske
		Aktion = Vorhersage der nächsten Aktion
		Führe Aktion in Spielumgebung aus
		Trage Werte in Historien ein
		Visualisiere Spielumgebung
	Plotte Historien
\end{lstlisting}
\end{minipage}