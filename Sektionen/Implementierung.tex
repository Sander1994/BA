\section{Implementierung}
In diesem Kapitel wird die Implementierung des Projektes vorgestellt und erläutert. Zur Darstellung der Funktionsweise der einzelnen Funktionen wird Pseudocode verwendet. Dies soll dabei helfen die Funktionsweise verständlicher und knapper aufzuzeigen.
\subsection{Einschränkungen}
Aus zeitlichen Gründen ergeben sich einige Einschränkungen für das Design der Implementierung. Das Spiel endet nach der Wahl des Würfels nach dem dritten Wurf in der sechsten Runde. Somit kann der Extra-Wahl-Bonus kein letztes Mal benutzt werden und es gibt in der letzten Runde auch keine Wahl vom Silbertablett. Außerdem werden nicht die niedrigsten sechs Würfel des Wurfes auf das Silbertablett des Mitspielers gelegt, sondern drei zufällige.
\subsection{Spielumgebung}
Die Implementierung besteht aus zwei Klassen. Eine davon ist die Spielumgebung. Zunächst werden in diesem Kapitel die wesentlichen Attribute der Spielumgebung und anschließend ihre Methoden erläutert. Für die Implementierung der Spielumgebung wurde die Bibliothek Gymnasium [siehe Unterabschnitt 2.2.1] verwendet.
\subsubsection{Klassenattribute}
\begin{minipage}{\linewidth}
Code 1 zeigt die Klassenattribute, die für das Rundenmanagement wichtig sind:
\vspace{0.5cm}
\begin{lstlisting}[caption={Klassenattribute für das Runden-System}, basicstyle=\ttfamily]
self.initial_rounds
self.rounds
self.roll_in_round
\end{lstlisting}
\end{minipage}

Das Attribut \texttt{initial\_rounds} beschreibt die maximale Rundenanzahl des Spiels und wird beim Zurücksetzen der Umgebung verwendet, um die Rundenzahl auf den gewünschten Wert (im Solo-Spiel sechs) zurückzusetzen.

Das Attribut \texttt{rounds} repräsentiert die aktuell verbleibende Rundenanzahl im Spiel.

Das Attribut \texttt{roll\_in\_round} repräsentiert die Nummer des aktuellen Wurfes in der Runde.\\

\begin{minipage}{\linewidth}
Code 2 zeigt die Klassenattribute, die für das Würfeln der Würfel relevant sind:
\vspace{0.5cm}
\begin{lstlisting}[caption={Klassenattribute für Würfel}, basicstyle=\ttfamily]
self.invalid_dice = {"white": False, "yellow": False, ...}
self.dice = {"white": 0, "yellow": 0, ...}
\end{lstlisting}
\end{minipage}

Die Attribute \texttt{invalid\_dice} und dice repräsentieren die Augenzahlen der Würfel, sowie die Gültigkeit der Würfel selbst. Ist der Wert von \texttt{invalid\_dice} False, ist der Würfel nicht ungültig und somit gültig.\\

\begin{minipage}{\linewidth}
Code 3 zeigt die Klassenattribute, die für das bilden einer Punktestandhistorie relevant sind:
\vspace{0.5cm}
\begin{lstlisting}[caption={Klassenattribute für die Nachvollziehbarkeit von Punkteständen}, basicstyle=\ttfamily]
self.score
self.score_history
self.initialized
\end{lstlisting}
\end{minipage}

Das Attribut \texttt{score} repräsentiert den aktuellen Punktestand der Spielumgebung. Es wird verwendet, um erreichte Punktestände in die \texttt{score\_history} einzutragen. Das Attribut \texttt{score\_history} ist eine Historie über die erreichten Punktestände in den einzelnen Episoden beziehungsweise Spieldurchläufen. Das Attribut \texttt{initialized} wird verwendet, um zu gewährleisten, dass nur Einträge in der Punktestandhistorie eingetragen werden, nachdem ein Spiel abgeschlossen wurde. Es trifft eine Aussage darüber ob die Spielumgebung bereits einmal initialisiert wurde oder nicht.\\

\begin{minipage}{\linewidth}
Code 4 zeigt die Klassenattribute, welche die farbigen Felder des Spielbrettes repräsentieren:
\vspace{0.5cm}
\begin{lstlisting}[caption={Klassenattribute für die farbigen Felder des Spiels}, basicstyle=\ttfamily]
self.yellow_field = [[3, 6, 5, 0], [2, 1, 0, 5], ...]
self.blue_field = [[0, 2, 3, 4], [5, 6, 7, 8], [9, 10, 11, 12]]
self.green_field = [0] * 11
self.orange_field = [0] * 11
self.purple_field = [0] * 11
\end{lstlisting}
\end{minipage}

Die Attribute \texttt{yellow\_field}, \texttt{blue\_field}, \texttt{green\_field}, \texttt{orange\_field} und \texttt{purple\_field} stehen für die fünf farbigen Felder auf dem Spielbrett. Sie repräsentieren die eingetragenen Werte auf dem Spielbrett und bestimmen somit welche Kästchen aktuell ausgefüllt werden können (vorausgesetzt die Würfelergebnisse passen) und welche Belohnungen freigeschaltet werden.\\

\begin{minipage}{\linewidth}
Code 5 zeigt die Klassenattribute, welche die zu erspielenden Boni auf den farbigen Feldern repräsentieren:
\vspace{0.5cm}
\begin{lstlisting}[caption={Klassenattribute für freizuschaltende Boni}, basicstyle=\ttfamily]
self.yellow_rewards = {"row": ["blue_cross", ...], "dia": ...}
self.blue_rewards = {"row": ["orange_five", ...], "col": ...}
self.green_rewards = [None, None, None, "extra_pick", ...]
self.orange_rewards = [None, None, "re_roll", ...]
self.purple_rewards = [None, None, "re_roll", ...]
\end{lstlisting}
\end{minipage}

Die Attribute \texttt{yellow\_rewards}, \texttt{blue\_rewards}, \texttt{green\_rewards}, \texttt{orange\_rewards} und \texttt{purple\_rewards} repräsentieren die freizuschaltenden Boni für die jeweiligen farbigen Felder. Für das blaue Feld sind diese in Form von Reihen (\texttt{row}) und Spalten (\texttt{col}) aufgeführt. Das gelbe Feld besitzt Boni für das Ausfüllen von Reihen (\texttt{row}) und einen Boni, der bei diagonalem Ausfüllen (\texttt{dia}) freigeschaltet werden kann. Für die Farben grün, orange und lila sind die Boni jeweils direkt einem der Kästchen im Feld zugewiesen, wobei viele der Kästchen keinen freizuschaltenden Bonus aufweisen, was dem Wert None entspricht.\\

\begin{minipage}{\linewidth}
Code 6 zeigt die Klassenattribute, für die Punktebelohnungen des gelben, blauen und grünen Feldes:
\vspace{0.5cm}
\begin{lstlisting}[caption={Klassenattribute für freizuschaltende Punktebelohnungen des gelben, blauen und grünen Feldes}, basicstyle=\ttfamily]
self.yellow_rewards = {"col": [10, 14, 16, 20], ...}
self.blue_count_rewards = [0, 1, 1, 2, 3, 4, 5, 6, 7, 8, 9, 10]
self.green_count_rewards = [1, 2, 3, 4, 5, 6, 7, 8, 9, 10, 11]
\end{lstlisting}
\end{minipage}

Im Attribut \texttt{yellow\_rewards} sind die Punktebelohnungen im Spaltenbereich (\texttt{col}) des gelben Feldes aufgeführt. Die Attribute \texttt{blue\_count\_rewards} und \texttt{green\_count\_rewards} repräsentieren die Punktebelohnungen, welche erspielt werden können, sobald ein blaues beziehungsweise grünes Kästchen ausgefüllt wird. Beginnend vom ersten Wert des Arrays und danach inkrementell aufsteigend, steigt die erhaltene Punktebelohnung bei jedem ausgefüllten Kästchen stetig an.\\

\begin{minipage}{\linewidth}
Code 7 zeigt die Klassenattribute der Flags für die Belohnungen der farbigen Felder. Sind diese gesetzt und sind die entsprechenden Kästchen für die Belohnung ausgefüllt, wurde die Belohnung bereits ausgeschüttet und wird es nicht erneut, bis die Attribute nach dem Spiel zurückgesetzt werden:
\vspace{0.5cm}
\begin{lstlisting}[caption={Klassenattribute für Belohnungsflags}, basicstyle=\ttfamily]
self.yellow_reward_flags = {"row": [False] * 4, "col": ...}
self.blue_reward_flags = {"row": [False] * 3, "col": ...}
self.blue_count_reward_flags = [False] * 12
self.green_reward_flags = [False] * 11
self.orange_reward_flags = [False] * 11
self.purple_reward_flags = [False] * 11
\end{lstlisting}
\end{minipage}

\begin{minipage}{\linewidth}
Code 8 zeigt die Klassenattribute für Punktestände der einzelnen farbigen Felder. Die Punktewerte werden aufaddiert, sobald Punkte im entsprechenden Feld erspielt worden sind und am Ende genutzt, um den Wert der Fuchs-Boni zu bestimmen [siehe Unterabschnitt 2.1.1]:
\vspace{0.5cm}
\begin{lstlisting}[caption={Klassenattribute für erreichte Punktestände der einzelnen farbigen Felder}, basicstyle=\ttfamily]
self.yellow_field_score
self.blue_field_score
self.green_field_score
self.orange_field_score
self.purple_field_score
\end{lstlisting}
\end{minipage}

\begin{minipage}{\linewidth}
Code 9 zeigt die Klassenattribute für freigeschaltete Boni. Wird eine Boni erspielt, wird ihr Wert inkrementiert, wird sie genutzt, wird er dekrementiert:
\vspace{0.5cm}
\begin{lstlisting}[caption={Klassenattribute für freigespielte Boni}, basicstyle=\ttfamily]
self.extra_pick
self.re_roll
self.fox
self.yellow_cross
self.blue_cross
self.green_cross
self.orange_four
self.orange_five
self.orange_six
self.purple_six
\end{lstlisting}
\end{minipage}

\begin{minipage}{\linewidth}
Code 10 zeigt die Klassenattribute für die Anzahl an gewählten Kästchen in den verschiedenen farbigen Feldern. Wenn ein Kästchen in einem der Felder ausgefüllt wird, wird der Wert des entsprechenden Attributes inkrementiert. Diese Attribute dienen nicht dem Spielablauf selbst, sondern der Nachvollziehbarkeit der Strategie des Modells:
\vspace{0.5cm}
\begin{lstlisting}[caption={Klassenattribute für die Anzahl an gewählte Kästchen innerhalb der farbigen Feldern}, basicstyle=\ttfamily]
self.picked_yellow
self.picked_blue
self.picked_green
self.picked_orange
self.picked_purple
\end{lstlisting}
\end{minipage}

\begin{minipage}{\linewidth}
Code 11 zeigt die Klassenattribute für den Aktions- sowie Beobachtungsraum:
\vspace{0.5cm}
\begin{lstlisting}[caption={Klassenattribute des Aktions- und Beobachtungsraumes}, basicstyle=\ttfamily]
self.number_of_actions = 247
low_bound = np.array([0]*16 + [0]*12 + ...)
high_bound = np.array([6]*16 + [6]*12 + ...)
self.action_space = spaces.Discrete(self.number_of_actions)
self.observation_space = spaces.Box(low_bound, high_bound, ...)
self.valid_action_mask_value = np.ones(self.number_of_actions)
\end{lstlisting}
\end{minipage}

Das Attribut \texttt{number\_of\_actions} repräsentiert die Gesamtanzahl an möglichen Aktionen des Modells. Die Attribute \texttt{low\_bound} und \texttt{high\_bound} setzen die obere und untere Grenze von Werten im Beobachtungsraum fest. Beispielsweise steht der erste Eintrag in beiden für Werte des gelben Feldes. Diese können von null \texttt{([0]*16)} bis sechs \texttt{([6]*16)} reichen. Das Attribut \texttt{action\_space} repräsentiert den Aktionsraum des Modells. Es ist ein diskreter Raum mit \texttt{number\_of\_actions} Werten von null bis \texttt{number\_of\_actions} minus eins. Das Attribut \texttt{observation\_space} repräsentiert den Beobachtungsraum. \texttt{Shape} definiert dabei die Größe des Beobachtungsraumes. Das Attribut \texttt{valid\_action\_mask\_value} repräsentiert die Aktionsmaske. Initial handelt es sich dabei um ein Numpy Array aus Einsen. Einsen stehen für gültige Aktionen, Nullen für ungültige. Die Werte verlaufen parallel zu den Werten des Aktionsraumes. Somit repräsentieren alle Werte (beispielsweise \texttt{[0]} oder \texttt{[5]}) sowohl im Aktionsraum als auch bei der Aktionsmaske die selbe Aktion.\\

Die Struktur des Aktionsraumes ist für das Verständnis der Arbeit wichtig, daher wird sie hier erläutert:

Die ersten 122 Werte des Aktionsraumes von 0 bis 121 sind sowohl normalen Wahlen bei eigenen Würfen als auch den verschiedenen Boni, welche es ermöglichen direkt ein Kästchen eines Felder anzukreuzen, zugeordnet. Dabei stehen die ersten 16 Werte für Kästchen des gelben Feldes, die nächsten 12 für Kästchen des blauen Feldes, und die folgenden 33 zu einer Aufteilung von jeweils 11 für Kästchen im grünen, orangenen und lila Feld. Die Werte von 61 bis 121 stehen für die selben Kästchen in der selben Reihenfolge wie die vorherigen 61 Werte, allerdings wird bei diesen der weiße Würfel verwendet statt des jeweils farbigen Würfels für das spezifische Feld.

Die Werte von 122 bis 243 stehen für Wahlen mit der Extra-Wahl-Boni oder für Wahlen vom Silbertablett des Gegners. Die Struktur innerhalb dieser Reichweite ist die selbe wie bei den 122 Werte zuvor. Die ersten 16 Werte stehen für die selben Kästchen im gelben Feld und so weiter.

Der Wert 244 steht für die Neu Würfeln Boni, der Wert 245 für das Passen bei einem möglichen Einsatz von Extra-Wahl-Boni und der Wert 246 für eine ungültige Aktion, die nur möglich ist, wenn keine der anderen Aktionen getätigt werden kann.
\subsubsection{Schritt-Methode}
Code 12 zeigt die Funktionsweise der Schritt-Methode (\texttt{step method}) der Spielumgebung mithilfe von Pseudocode. Diese Methode führt Spielschritte beziehungsweise Aktionen in der Spielumgebung aus:
\vspace{0.5cm}
\begin{lstlisting}[caption={Schritt-Methode}]
step(Aktion):
	Setze Methodenattribute zurück
	
	if not Extra-Wahl:
		if Aktion im gelben Feld:
			Fülle entsprechendes gelbes Kästchen aus
			if Gelbes-Kreuz-Boni >= 1
				Bonusrunde = True
				Gelbes-Kreuz-Boni -= 1
			if not Bonusrunde:
				Setze entsprechende Würfel auf ungültig
		if Aktion im blauen Feld:
			Fülle entsprechendes blaues Kästchen aus
			if Blaues-Kreuz-Boni >= 1:
				Bonusrunde = True
				Blaues-Kreuz-Boni -= 1
			if not Bonusrunde:
				Setze entsprechende Würfel auf ungültig
		if Aktion im grünen Feld:
			Fülle entsprechendes grünes Kästchen aus
			if Grünes-Kreuz-Boni >= 1:
				Bonusrunde = True
				Grünes-Kreuz-Boni -= 1
			if not Bonusrunde:
				Setze entsprechende Würfel auf ungültig
		if Aktion im orangene Feld:
			Fülle entsprechendes orangenes Kästchen aus
			if Orangene-Feld-Boni >= 1:
				Bonusrunde = True
				Orangene-Feld-Boni -= 1
			if not Bonusrunde:
				Setze entsprechende Würfel auf ungültig
		if Aktion im lila Feld:
			Fülle entsprechendes lila Kästchen aus
			if Lila-Sechs-Boni >= 1:
				Bonusrunde = True
				Lila-Sechs-Boni -= 1
			if not Bonusrunde:
				Setze entsprechende Würfel auf ungültig
	
	if Extra-Wahl:
		Bonusrunde = True
		if Extra-Wahl-Bonus benutzt:
			Extra-Wahl-Boni -= 1
		Finde Kästchen zum ausfüllen und fülle es aus
		Setze gewählten Würfel auf ungültig
		if Extra-Wahl-Boni <= 0 or Wahl erfolgte vom Silbertablett:
			Runde wird inkrementiert
			
	if Neu-Würfeln-Bonus wird benutzt:
		Werfe Würfel neu
		Neu-Würfeln-Boni -= 1
		Aktualisiere Aktionsmaske
		return Beobachtungsraum, Punktebelohnung, Spiel terminiert?
		
	if Statt Extra-Wahl-Boni gepasst wird:
		Runde wird inkrementiert
		return Beobachtungsraum, Punktebelohnung, Spiel terminiert?
	if ungültige Aktion gewählt da keine gültigen Aktionen vorhanden:
		Nichts tun
			
	Punktebelohnung += erspielte Belohnung in diesem Schritt
	if not Bonusrunde:
		Runde wird inkrementiert
		if Rundenanzahl == 0:
			terminiert = True
			Punktebelohnung += Fuchsbonipunktebelohnung
	Aktualisiere Aktionsmaske
	return Beobachtungsraum, Punktebelohnung, Spiel terminiert?		
\end{lstlisting}
\subsubsection{Methode zum Zurücksetzen der Spielumgebung}
\begin{minipage}{\linewidth}
Code 13 zeigt die Funktionsweise der Methode zum Zurücksetzen der Spielumgebung (\texttt{reset method}) mithilfe von Pseudocode. Sie setzt Umgebungen, in denen ein Spiel abgeschlossen wurde, auf den Anfangszustand zurück, damit eine weitere Runde gespielt werden kann:
\vspace{0.5cm}
\begin{lstlisting}[caption={Methode zum Zurücksetzen der Umgebung}]
reset():
	Setze alle Attribute für den Spielablauf auf den Startzustand
\end{lstlisting}
\end{minipage}

\subsubsection{Methode zur Visualisierung der Spielumgebung}
\begin{minipage}{\linewidth}
Code 14 zeigt die Funktionsweise der Funktion zur Visualisierung der Spielumgebung (\texttt{render method}) mithilfe von Pseudocode. Die Visualisierung dient einer verbesserten Nachvollziehbarkeit der Ereignisse innerhalb der Spielumgebung:
\vspace{0.5cm}
\begin{lstlisting}[caption={Methode zur Visualisierung der Spielumgebung}]
render():
	Zeige alle relevanten Attribute und Merkmale der Umgebung an
\end{lstlisting}
\end{minipage}

\subsubsection{Würfel-Methode}
\begin{minipage}{\linewidth}
Code 15 zeigt die Funktionsweise der Methode zum Werfen der Würfel (\texttt{roll\_dice method}) mithilfe von Pseudocode. Würfel müssen nach jedem Wurf, bei Anfang jeder Spielrunde und beim Einsetzen des Neu-Würfeln-Bonus neu geworfen werden:
\vspace{0.5cm}
\begin{lstlisting}[caption={Methode zum Werfen der Würfel}]
roll_dice():
	Werfe Würfel neu
\end{lstlisting}
\end{minipage}

\subsubsection{Methode zur Überprüfung der freigespielten Belohnungen}
\begin{minipage}{\linewidth}
Code 16 zeigt die Funktionsweise der Methode zur Überprüfung der freigespielten Belohnungen (\texttt{check\_rewards method}) mithilfe von Pseudocode. Diese Methode überprüft jedes Mal, nachdem ein Kästchen in der Spielumgebung ausgefüllt worden ist, ob und welche Belohnung freigespielt wurde und fügt diese dem Inventar des Spielers hinzu:
\vspace{0.5cm}
\begin{lstlisting}[caption={Methode zur Überprüfung der freigespielten Belohnungen}]
check_rewards():
	Überprüfe ob im gelben Feld Belohnungen freigeschaltet wurden
	Schalte im gelben Feld freigeschaltete Belohnungen frei
	
	Überprüfe ob im blauen Feld Belohnungen freigeschaltet wurden
	Schalte im blauen Feld freigeschaltete Belohnungen frei
	
	Überprüfe ob im grünen Feld Belohnungen freigeschaltet wurden
	Schalte im grünen Feld freigeschaltete Belohnungen frei
	
	Überprüfe ob im orangenen Feld Belohnungen freigeschaltet wurden
	Schalte im orangenen Feld freigeschaltete Belohnungen frei
	
	Überprüfe ob im lila Feld Belohnungen freigeschaltet wurden
	Schalte im lila Feld freigeschaltete Belohnungen frei
	
	return Punktebelohnung
\end{lstlisting}
\end{minipage}

\subsubsection{Methode zur Generierung des Beobachtungsraumes}
\begin{minipage}{\linewidth}
Code 17 zeigt die Funktionsweise der Methode zur Generierung des Beobachtungsraumes (\texttt{\_get\_obs method}) mithilfe von Pseudocode. Diese Methode vereinfacht es, den Beobachtungsraum zu generieren, welcher jedes Mal benötigt wird, wenn ein Schritt in der Umgebung ausgeführt wird:
\vspace{0.5cm}
\begin{lstlisting}[caption={Methode zur Generierung des Beobachtungsraumes}]
_get_obs():
	Erstelle Numpy Arrays für farbige Felder
	Erstelle Numpy Array für Würfelergebnisse
	Erstelle Numpy Array für ungültige Würfel
	Füge Arraywerte für Boni und Rundenzahl hinzu
	Beobachtungsraum = Verbinde alle Arrays miteinander
	
	return Beobachtungsraum
\end{lstlisting}
\end{minipage}

\subsubsection{Methode zur Generierung der Aktionsmaske}
Code 18 zeigt Die Funktionsweise der Methode zur Generierung der Aktionsmaske (\texttt{valid\_action\_mask method}) mithilfe von Pseudocode. Einsen stehen für gültige Aktionen und Nullen für ungültige. Die Wahlwahrscheinlichkeit des Modells für ungültige Aktionen wird auf null gesetzt:
\vspace{0.5cm}
\begin{lstlisting}[caption={Methode zur Generierung der Aktionsmaske}]
valid_action_mask():
	Setze alle Werte auf Eins
	
	Werte für bereits ausgefüllte gelbe Kästchen = 0
	Werte für gelbe Kästchen ohne passende Würfelergebnisse = 0
	
	Werte für bereits ausgefüllte blaue Kästchen = 0
	Werte für blaue Kästchen ohne passende Würfelergebnisse = 0
	
	Werte für bereits ausgefüllte grüne Kästchen = 0
	Werte für grüne Kästchen ohne passende Würfelergebnisse = 0
	
	Werte für bereits ausgefüllte orangene Kästchen = 0
	Werte für orangene Kästchen ohne passende Würfelergebnisse = 0
	
	Werte für bereits ausgefüllte lila Kästchen = 0
	Werte für lila Kästchen ohne passende Würfelergebnisse = 0
	
	Werte für Akionen mithilfe ungültiger Würfel = 0
	
	if not Extra-Wahl or ungültige Extra-Wahl:
		Werte von 122 bis 243 & Wert für das Aussetzen = 0
	if gültige Extra-Wahl:
		Werte von 0 bis 121 = 0
		Wert für das Aussetzen = 1
	
	if Neu-Würfeln-Boni <= 0 or Extra-Wahl:
		Wert für Neu-Würfeln-Bonus = 0
	
	if einer der Boni zum direkten Ankreuzen von Kästchen >= 1:
		Alle Werte = 0
	
	if Gelbes-Kreuz >= 1:
		Werte von 0 bis 15 = 1
	if Blaues-Kreuz >= 1:
		Werte von 16 bis 27 = 1
	if Grünes-Kreuz >= 1:
		Werte von 28 bis 38 = 1
	if (Orangene-Vier or Orangene-Fünf or Orangene-Sechs) >= 1:
		Werte von 39 bis 49 = 1
	if Lila-Sechs >= 1:
		Werte von 50 bis 60 = 1
	
	if Alle Werte außer 246 == 0:
		Wert für ungültige Aktion = 1
	else:
		Wert für ungültige Aktion = 0
		
	return Aktionsmaske
\end{lstlisting}
\subsubsection{Methode zum Hinzufügen von Boni}
\begin{minipage}{\linewidth}
Code 19 zeigt die Funktionsweise der Methode zum Hinzufügen freigeschalteter Boni (\texttt{add\_reward method}) mithilfe von Pseudocode. Diese Methode wird von der Methode zur Überprüfung der freigespielten Belohnungen genutzt, um Boni dem Inventar des Spielers hinzuzufügen:
\vspace{0.5cm}
\begin{lstlisting}[caption={Methode zum Hinzufügen freigschalteter Boni}]
add_reward(Belohnungstyp):
	Inkrementiere Wert für Belohnungstyp
\end{lstlisting}
\end{minipage}

\subsubsection{Methode zum Inkrementieren von Runden}
Code 20 zeigt die Funktionsweise der Methode zum Inkrementieren des Runden-Systems (\texttt{increment\_rounds method}) mithilfe von Pseudocode. Diese Methode ist dafür zuständig das Runden-System des Spiels voranzutreiben:
\vspace{0.5cm}
\begin{lstlisting}[caption={Methode zum Inkrementieren des Runden-Systems}]
increment_rounds():
	if Extra-Wahl nach eignener Runde:
		Wahl vom Silbertablett = True
		Extra Wahl nach eigener Runde = False
		Setze alle Würfel auf gültig
		Würfele Würfel neu
		Setze drei zufällige Würfel auf ungültig
		
	elif Wahl vom Silbertablett:
		if Extra-Wahl-Boni >= 1:
			Extra-Wahl nach Wahl vom Silbertablett = True
		else:
			Schalte Boni für erreichte Runde frei
			Werfe Würfel neu
		Wahl vom Silbertablett = False
		Setze alle Würfel auf gültig
	
	elif Extra-Wahl nach Wahl vom Siblertablett:
		Extra-Wahl nach Wahl vom Silbertablett = False
		Werfe Würfel neu
		Setze alle Würfel auf gültig
		Schalte Boni für erreichte Runde frei
	
	elif Wurf in Runde >= 3:
		Rundenanzahl -= 1
		Wurf in Runde = 1
		Setze alle Würfel auf gültig
		if Extra-Wahl-Boni >= 1:
			Extra Wahl nach eigener Runde = True
		else:
			Wahl vom Silbertablett = True
			Werfe Würfel neu
			Setze drei zufällige Würfel auf ungültig
	else:
		Wurf in Runde += 1
		Werfe Würfel neu
\end{lstlisting}
\subsection{Künstliche Intelligenz}
Die Implementierung besteht aus zwei Klassen. Eine davon ist die Künstliche Intelligenz. In diesem Kapitel werden die Methoden der Künstlichen Intelligenz mithilfe von Pseudocode erläutert. Diese Methoden beinhaltet Methoden der Spielumgebung [siehe Abschnitt 4.1] und Methoden zur Visualisierung [siehe Abschnitt 4.3]. Für die Implementierung der Künstlichen Intelligenz die Bibliothek Stable Baselines [siehe Unterabschnitt 2.2.2] verwendet.
\subsubsection{Methode zum Anlernen des Modells}
\begin{minipage}{\linewidth}
Code 21 zeigt die Funktionsweise der Methode zum Anlernen des Modells (\texttt{model\_learn method}) mithilfe von Pseudocode. Es werden Hyperparameter festgelegt, welche genutzt werden, um ein MaskablePPO-Modell zu generieren und mithilfe der Spielumgebung zu trainieren. Alternativ kann ein bereits trainiertes Modell geladen und weiter trainiert werden:
\vspace{0.5cm}
\begin{lstlisting}[caption={Methode zu Anlernen des Modells}]
model_learn(Hyperparameter):
	Initialisiere Spielumgebungen
	Modell = Initialisiere MaskablePPO-Modell mit Hyperparamter
	if Modellname in Hyperparameter:
		Modell = Lade Modell mit Modellname
		Modell.Spielumgebungen = Initialisieren Spielumgebungen
		Setze Entropie-Koeffizient
	Lerne Modell an
	Setze Gamma für Vorhersagen
	Setze Entropie-Koeffizient für Vorhersagen
	Lerne Modell erneut an
	Speichere Modell
\end{lstlisting}
\end{minipage}
\subsubsection{Methode zum Vorhersagen mithilfe des Modells}
\begin{minipage}{\linewidth}
Code 22 zeigt die Funktionsweise der Methode zum Vorhersagen von Aktionen mithilfe des Modells (\texttt{model\_predict method}) unter Verwendung von Pseudocode. Diese Methode ermöglicht es, mithilfe des Modells Vorhersagen über günstige Aktionen in einem gegebenen Zustand der Spielumgebung vorherzusagen und anhand dessen einen Spielverlauf zu simulieren:
\vspace{0.5cm}
\begin{lstlisting}[caption={Methode zum Vorhersagen von Aktionen mithilfe des Modells}]
model_predict(Schrittanzahl):
	Lade Modell
	Initialisiere Spielumgebungen und Historien
	Setze Beobachtungsraum
	for i in range(Schrittanzahl):
		Aktualisiere Aktionsmaske
		Aktion = Vorhersage der nächsten Aktion
		Führe Aktion in Spielumgebung aus
		Trage Werte in Historien ein
		Visualisiere Spielumgebung
	Plotte Historien
\end{lstlisting}
\end{minipage}
\subsubsection{Methode zur Initialisierung der Spielumgebungen}
\begin{minipage}{\linewidth}
Code 23 zeigt die Funktionsweise der Methode zur Initialisierung der Spielumgebungen (\texttt{\_init\_envs method}) mithilfe von Pseudocode. Mit dieser Methode werden Spielumgebungen initialisiert und einer Vektorumgebung zugewiesen. Diese Vektorumgebung ermöglicht es mehrere Spielumgebungen gleichzeitig zu bearbeiten. Außerdem werden Variablen für die Nachvollziehbarkeit der Abläufe innerhalb der Spielumgebung (Historien) initialisiert:
\vspace{0.5cm}
\begin{lstlisting}[caption={Methode zur Initialisierung der Spielumgebungen}]
_init_envs(Anzahl, Punktestände, Fehlversuche):
	_init():
		Spielumgebung = Initialisieren eine Spielumgebung
		Setze Aktionsmasker für die Spielumgebung
		return Spielumgebung
	Initialisiere Verktorumgebung(Anzahl, _init)
	if not Punktestände and not Fehlversuche:
		return Vektorumgebung
	if Punktestände and not Fehlversuche:
		Erstelle Variablen für Punkteständehistorie
	if not Punktestände and Fehlversuche:
		Erstelle Variablen für Fehlversuchehistorie
	if Punktestände and Fehlversuche:
		Erstelle Variablen für Punkteständehistorie
		Erstelle Variablen für Fehlversuchehistorie
	return Vektorumgebung, Variablen für Historien
\end{lstlisting}
\end{minipage}

\subsubsection{Methode zum Anwenden der Aktionsmaske}
\begin{minipage}{\linewidth}
Code 24 zeigt die Funktionsweise der Methode zum Anwenden der Aktionsmaske (\texttt{mask\_fn method}) mithilfe von Pseudocode. Die Methode dient dazu dem Modell die Aktionsmasken der einzelnen Spielumgebungen zu übergeben:
\vspace{0.5cm}
\begin{lstlisting}[caption={Methode zum Anwenden der Aktionsmaske}]
mask_fn(Spielumgebung):
	Caste Spielumgebung in benutzerdefinierte Spielumgebung
	return Aktionsmaske der Spielumgebung
\end{lstlisting}
\end{minipage}

\subsection{Darstellung}
Dieses Kapitel erläutert die Methoden zur Visualisierung mithilfe von Pseudocode. Für diese Methoden wurde die Bibliothek Matplotlib [siehe Unterabschnitt 2.2.3] verwendet.
\subsubsection{Methoden zum Erstellen von Einträgen}
\begin{minipage}{\linewidth}
Code 25 zeigt die Funktionsweise der Methoden zum Erstellen von Einträgen und Historien (\texttt{make\_fail\_entries method, make\_score\_entries method, make\_fail\_history\_entry method, make\_score\_history\_entry method}) mithilfe von Pseudocode. Diese Methoden erstellen Einträge und Historien von erzielten Punkteständen und der Anzahl getätigter ungültiger Aktionen innerhalb abgeschlossener Spiele:
\vspace{0.5cm}
\begin{lstlisting}[caption={Methoden zum Erstellen von Einträgen für Historien}]
make_fail_entries(Punktebelohnungen, Anzahl, Fehlversuche):
	if Punktebelohnung < 0:
		Inkrementiere Fehlversuche der Umgebung

make_score_entries(Punktebelohnungen, Anzahl, Punktestände):
	if Punktebelohnung > 0:
		Addiere Punktebelohnung zum Punktestand der Umgebung

make_fail_history_entry(Fehlversuche, Fehlversuchshistorie)
	if Umgebung terminiert:
		Hänge Fehlversuche an Fehlversuchshistorie an
		
make_score_history_entry(Punktestände, Punktehistorie)
	if Umgebung terminiert:
		Hänge Punktestand der Umgebung Punktehistorie an
\end{lstlisting}
\end{minipage}
\subsubsection{Methode zum Plotten von Historien}
\begin{minipage}{\linewidth}
Code 26 zeigt die Funktionsweise der Methode zum Plotten von Historien (\texttt{plot\_history method}) mithilfe von Pseudocode. Diese Methode visualisiert die generierten Historien:
\vspace{0.5cm}
\begin{lstlisting}[caption={Methode zum Plotten von Historien}]
plot_history(Historie):
	Plotte jeden Eintrag der Historie
	Setze Titel
	Setze Labels
	Zeige Grafik an
\end{lstlisting}
\end{minipage}
\subsection{Anwendungsbeispiel für die Künstliche Intelligenz}
\begin{minipage}{\linewidth}
Code 27 zeigt ein Anwendungsbeispiel des Projektes. Es wird ein Modell mit dem Namen \texttt{model\_name} geladen und für 1110000 Schritte (x3 durch die \texttt{learn\_model methode}) trainiert. Entsprechende Hyperparameter werden ebenfalls gesetzt. Daraufhin werden mithilfe des Modells Spieldurchläufe (für ungefähr 1000 Episoden) simuliert:
\vspace{0.5cm}
\begin{lstlisting}[caption={Anwendungsbeispiel für die Künstliche Intelligenz}]
def main():
	model_learn(total_timesteps=1110000, ent_coef=0.1, gamma=1, 
	 model_name="maskableppo_ganzschoenclever_193avg_v3")
	
	model_predict(n_envs=1, render=True, n_steps=40000,
	 model_name="maskableppo_ganzschoenclever")
\end{lstlisting}
\end{minipage}