\section{Zusammenfassung und Ausblick}

Insgesamt lässt sich sagen, dass das Projekt ein Erfolg war. Es ließ sich eine Künstliche Intelligenz mithilfe relativ einfacher Methoden für das Spiel Ganz schön clever implementieren und diese erzielte sehr gute Ergebnisse. 

Besonders hilfreich war es einen Prototypen zu bauen und von diesem Punkt an weiterzuarbeiten. Dies half dabei leichter einen Überblick über die wichtigsten Aspekte zu gewinnen und diese auch umzusetzen, statt ausschweifend zu planen. 

Die Implementierung der Künstlichen Intelligenz selbst gestaltete sich dank Bibliotheken wie Gymnasium und Stable Baselines vergleichsweise einfach, brachte aber dennoch eine Vielzahl interessanter und nützlicher Erkenntnisse mit sich. Besonders hervorzuheben ist dabei die Anpassung von Hyperparametern, welche zum Teil zu sehr unterschiedlichen Ergebnissen geführt hat. 

Der Aufwändigste Teil des Projektes war die Implementierung des Spielumgebung selbst. Das Spiel hat viele kleine Interaktionen, welche zum Teil ineinander verschachtelt sind und es gestaltete sich zum Teil schwierig alle Zusammenhänge zu überschauen und zu gewährleisten, dass Änderungen an einer Funktion andere nicht negativ beeinflusst. Es sind auch vielerlei Bugs aufgetreten, welche gefunden und behoben werden mussten. Der gravierendste Fehler war es das Runden-System nicht von vornherein zu implementieren, sondern es später einzufügen, als die meisten anderen Aspekte des Spiels bereits implementiert waren. Dies führte dazu, dass viele Aspekte der Spielumgebung neu überdacht und überarbeitet werden mussten. Für die Zukunft bleibt zu sagen, dass man sich auf alle grundlegenden Aspekte des Spiels/Problems konzentrieren und diese lauffähig machen sollte, bevor man anfängt bereits Funktionierende Aspekte zu erweitern. Der Einfluss einer Änderung des Runden-Systems wurde unterschätzt.\\

Für die Weiterarbeit am Projekt schlage ich vor, eine grafische Visualisierung des Lern- oder Vorhersageprozesses zu implementieren. Außerdem lässt sich das Modell sicherlich mithilfe anderen Algorithmen, Trainingsverfahren oder Hyperparameter weiter optimieren. Was im Rahmen der Arbeit ebenfalls nur bedingt erfolgt ist, ist eine Analyse der Spielstrategien von Modellen und wie sich unterschiedliche Strategien auf die Performance auswirken. Zwar wurden Variablen hinzugefügt, welche messen, wie oft Kästchen in bestimmten Feldern ausgefüllt wurden, allerdings lässt sich hier noch viel mehr machen. Ebenfalls wäre es sinnvoll Tests für das Projekt zu implementieren, um sicherzustellen, dass alle Aspekte der Umgebung wie vorhergesehen funktionieren und zusammenarbeiten. Zwar erfolgten Tests durch die Auswertung von Umgebungsvariablen und Print-Statements, aber fehlt dem Projekt eine fundierte Umsetzung professioneller Testverfahren.