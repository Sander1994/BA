\section{Einleitung}
\subsection{Hinführung zum Thema}
In den vergangenen Jahren gewann Künstliche Intelligenz (KI) und insbesondere maschinelles Lernen mit steigender Tendenz an Bedeutung \cite{statista_aiworldwide}. In der Spielentwicklung nehmen Künstliche Intelligenz und maschinelles Lernen schon seit Langem eine zentrale Rolle ein \cite{noauthor_kunstliche_2023,millington_ai_2019}. In vielen Spielen gibt es eine oder mehrere Künstliche Intelligenzen, die bestimmte Aufgaben erfüllen, um den Spieler im Spiel zu unterstützen oder ihm als Widersacher zu dienen. Je komplexer die Aufgabenstellung desto schwieriger ist es, eine Künstliche Intelligenz zu generieren, welche diese effizient und richtig lösen kann. 

Ein bedeutender Zweig innerhalb der Künstlichen Intelligenz ist das Reinforcement Learning (RL). Reinforcement Learning bestärkt vorteilhafte Aktionen einer KI, um ihr ein gewünschtes Verhalten beizubringen \cite[S. 11]{ris-ala_fundamentals_2023}. Forschung im Bereich des Reinforcement Learning in strategischen Brettspielen ist besonders interessant, da ihre Komplexität mit realen Aufgaben vergleichbar ist und ihr Testumfeld den Vergleich vieler verschiedener Spieler sowie KI-Techniken ermöglicht \cite[S. 1]{millington_ai_2019}. Eines der bekanntesten Beispiele für den Einsatz von Deep Reinforcement Learning ist AlphaGo.

Das Brettspiel \textit{"Ganz schön clever"} ist ein Würfelspiel, das eine hohe Komplexität aufweist. Diese Komplexität kommt vor allem durch die hohe Anzahl an möglichen Aktionen, die vielen Zusammenhänge des Belohnungssystems und die hohe Stochastizität des Spiels zustande. Interessant ist, wie man für ein solches komplexes Spiel eine Künstliche Intelligenz entwickeln kann. Dabei könnte die Komplexität des Spiels möglicherweise zu groß sein, um von der Künstlichen Intelligenz erfasst zu werden und es können sich zahlreiche weitere Problemstellungen ergeben, für die Lösungen gefunden werden müssen.

Eines der bedeutendsten neuen Tools im Jahr 2023 ist ChatGPT 4. Dieses Tool ist ein Chat-Bot, der es dem Benutzer ermöglicht mit ihm zu kommunizieren und ihm Fragen oder Aufgaben zu stellen. Solche Tools ermöglichen es Benutzern zunehmend, ihre Tätigkeiten zu vereinfachen und prägen somit das Leben der Menschen zunehmend \cite{tagesschaude_chatgpt_nodate}. Im Rahmen dieser Arbeit wurde ChatGPT 4 als unterstützendes Tool verwendet. Vor allem wurde ChatGPT 4 zur Beantwortung fachlicher Fragen und anfänglich zur Generierung von Code für einen Prototypen genutzt.

\newpage
\subsection{Zielsetzung und Motivation}
Ziel der Arbeit ist es, eine Künstliche Intelligenz zu entwickeln, die das Spiel \textit{"Ganz schön clever"} möglichst gut spielen kann. Gut bedeutet in diesem Kontext, dass im Durchschnitt eine möglichst hohe Punktzahl innerhalb einer Spielrunde erzielt wird. 

Es soll analysiert werden, welche Aspekte es bei der Entwicklung zu beachten gilt und wie sich unterschiedliche Ansätze auf das Verhalten und die Performanz der Künstlichen Intelligenz auswirken. Da es noch keine Untersuchungen zu einer Spiel-KI für das Spiel \textit{"Ganz schön clever"} gibt, ist es besonders interessant Erkenntnisse darüber zu sammeln, welche Probleme sich bei der Entwicklung ergeben und wie diese gelöst werden können.
\subsection{Aufgabenstellung}
Es ist eine Künstliche Intelligenz für das Spiel \textit{"Ganz schön clever"} zu implementieren. Der Entwicklungsprozess und dessen Ergebnisse sollen analysiert und bewertet werden. Zunächst ein Prototyp entwickelt, welcher eines der fünf Felder des Spiels beinhaltet. Dieser Prototyp soll Einsichten über die Machbarkeit und die Rahmenbedingungen des Projektes bringen. Daraufhin werden die Künstliche Intelligenz und die Spielumgebung schrittweise um ihre jeweiligen Funktionalitäten erweitert, bis das Spiel vollständig und möglichst gut von der Künstlichen Intelligenz gespielt werden kann.

\newpage
\subsection{Aufbau der Arbeit}
Kapitel 2 enthält die Grundlagen der Arbeit. Zunächst wird das Spiel \textit{"Ganz schön clever"} mit seine Mechaniken erklärt. Anschließend werden Grundlagen für maschinelles Lernen, Reinforcement Learning, Deep Learning und Proximal Policy Optimization behandelt, um eine fachliche Grundlage für das Verständnis der Abläufe beim Training der Künstlichen Intelligenz zu bieten. Im Folgenden werden die verwendeten Technologien des Projektes vorgestellt. Die vorgestellten Technologien sind Gynmansium, Stable Baselines 3, Matplotlib und ChatGPT 4.

Kapitel 3 befasst sich mit den Anforderungen und der Konzeption des Projektes. Zuerst werden die Rahmenbedingungen des Projekts erläutert. Danach wird das Zusammenspiel aus Künstlicher Intelligenz und Spielumgebung erklärt. Daraufhin folgen die Anforderungen an die Implementierung des Spiels und der Künstlichen Intelligenz. Es folgt eine Erklärung des Zusammenspiels aus KI und Spielumgebung. Hierbei wird vor allem der Trainingsprozess veranschaulicht. Anschließend werden die Designs von Spielumgebung und Künstlicher Intelligenz vorgestellt.

Kapitel 4 zeigt und erklärt die Implementierung des Projektes. Dabei werden die wesentlichen Variablen und Methoden des Projektes erläutert und beschrieben. Für Methoden wird dabei Pseudocode verwendet, um die Implementierung einfacher verständlich zu machen und den Umfang zu begrenzen.

Kapitel 5 befasst sich mit den Ergebnissen des Projektes, wobei zunächst der Verlauf der Implementierungsphase und dessen einzelne Meilensteine behandelt werden. Daraufhin folgen die Analyse und Darstellung des finalen Modells.

Das abschließende Kapitel 6 beinhaltet eine Zusammenfassung der Arbeit und Anreize zur Weiterarbeit am Projekt.