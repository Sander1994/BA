\section{Einleitung}
\subsection{Hinführung zum Thema}
In den vergangenen Jahren gewann maschinelles Lernen und insbesondere die Künstliche Intelligenz zunehmend an Bedeutung, Tendenz steigend. Im Jahr 2023 ist eines der präsentesten neuen Tools ChatGPT 4. Dieses Tool ist ein Chat-Bot, welcher es dem Benutzer ermöglicht mit ihm zu kommunizieren und ihm Fragen oder Aufgaben zu stellen. Solche Tools ermöglichen es Benutzern zunehmend ihre Tätigkeiten zu vereinfachen und prägen somit das Leben der Menschen zunehmend. Auch in dieser Arbeit wurde ChatGPT 4 als unterstützendes Tool verwendet. Es wurde vor Allem dafür benutzt um fachliche Fragen zu beantworten, aber auch anfangs um Code für den Prototypen zu generieren. Mit zunehmender Komplexität der zu bearbeitenden Aufgabe sinkt die Verlässlichkeit solcher Tools. Daher ist es wichtig die Anfragen an den Chat-Bot möglichst präzise zu formulieren und den Aufgabenbereich angemessen einzuschränken um das Tool nicht zu überfordern.

Auch in der Spielentwicklung nimmt das maschinelle Lernen und die Künstliche Intelligenz schon seit langem eine zentrale Rolle ein. In den meisten Spielen gibt es eine oder mehrere Künstliche Intelligenzen, welche bestimmte Aufgaben erfüllen, um den Spieler bei Spiel zu unterstützen oder ihm als Widersacher zu dienen. Auch hier gilt je komplexer die Aufgabenstellung desto schwieriger ist es einen solchen Bot zu generieren, welcher diese effizient und richtig lösen kann.

Das Gemeinschaftsspiel "Ganz schön clever" ist ein Würfelspiel, welches eine hohe Komplexität aufweist. Diese kommt vor allem durch die hohe Anzahl an möglichen Aktionen (dem sogenannte Aktionsraum) für den Spieler und die vielen Zusammenhänge des Belohnungssystems im Spiel zu Stande. Das Spiel weist zusätzlich eine hohe Stochastizität auf, welche die Komplexität weiter erhöht.

Interessant ist wie man für ein solch komplexes Spiel einen Bot oder eine Künstliche Intelligenz entwickeln kann um dieses effizient spielen zu können. Ist die Komplexität möglicherweise zu groß, um vom Bot erfasst zu werden und wenn nicht, wie kann man einen solchen Bot implementieren und was gilt es dabei zu beachten?
\subsection{Zielsetzung und Motivation}
Ziel der Arbeit ist es einen Bot beziehungsweise eine Künstliche Intelligenz zu entwickeln, welche das Spiel "Ganz schön clever" möglichst effizient spielen kann. Dabei soll analysiert werden, welche Aspekte es dabei zu beachten gilt und wie sich unterschiedliche Ansätze auf das Verhalten und die Performance des Modells (des Bots) auswirken.

In den vergangenen Jahren hat sich viel getan, weshalb deutlich mehr möglich geworden ist. Mit neuen Möglichkeiten ergeben sich auch bessere oder einfachere Ansätze, die zu einem wünschenswerten Ergebnis führen. Ziel ist es auch einen geeigneten Ansatz zu finden und zu vervollständigen.

Es gibt des Weiteren noch keine Untersuchungen zu einer Spiel-KI für das Spiel "Ganz schön clever" daher ist es interessant Erkenntnisse darüber zu gewinnen welche Schwierigkeiten sich hierbei ergeben und wie man diese überwinden kann.
\subsection{Aufgabenstellung}
Es ist eine KI für das Spiel "Ganz schön clever" zu implementieren. Hierbei sollen der Vorgang sowie Ergebnisse des Prozesses analysiert und bewertet werden. Hierzu wird zunächst ein Prototyp entwickelt, welcher eines der fünf Felder des Spiels beinhaltet. Dieser soll Einsichten über die Machbarkeit und die Rahmenbedingungen des Projektes geben. Daraufhin werden das Modell und die Spielumgebung schrittweise um ihre jeweiligen Funktionalitäten erweitert, bis das Spiel vollständig und möglichst effizient von der Künstlichen Intelligenz gespielt werden kann.
\subsection{Aufbau der Arbeit}