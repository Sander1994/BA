\section{Einleitung}
\subsection{Hinführung zum Thema}
In den vergangenen Jahren gewann Künstliche Intelligenz und insbesondere maschinelles Lernen mit steigender Tendenz an Bedeutung \cite{noauthor_kunstliche_nodate,noauthor_maschinelles_2023}. Eines der präsentesten neuen Tools im Jahr 2023 ist ChatGPT 4. Dieses Tool ist ein Chat-Bot, welcher es dem Benutzer ermöglicht mit ihm zu kommunizieren und ihm Fragen oder Aufgaben zu stellen. Solche Tools ermöglichen es Benutzern zunehmend ihre Tätigkeiten zu vereinfachen und prägen somit das Leben der Menschen zunehmend \cite{tagesschaude_chatgpt_nodate}. In dieser Arbeit wurde ChatGPT 4 ebenfalls als unterstützendes Tool verwendet. Vor allem wurde ChatGPT 4 zur Beantwortung fachlicher Fragen und anfänglich zur Generierung von Code für den Prototypen genutzt. Mit zunehmender Komplexität der zu bearbeitenden Aufgabe sinkt die Verlässlichkeit solcher Tools. Daher ist es wichtig, die Anfragen an den Chat-Bot möglichst präzise zu formulieren und den Aufgabenbereich angemessen einzuschränken, um das Tool nicht zu überfordern.

Auch in der Spielentwicklung nehmen Künstliche Intelligenz und maschinelles Lernen schon seit langem eine zentrale Rolle ein \cite{noauthor_kunstliche_2023}. In vielen Spielen gibt es eine oder mehrere Künstliche Intelligenzen, welche bestimmte Aufgaben erfüllen, um den Spieler bei Spiel zu unterstützen oder ihm als Widersacher zu dienen. Auch hier gilt je komplexer die Aufgabenstellung desto schwieriger ist es, einen solchen Bot zu generieren, welcher diese gut und richtig lösen kann.

Das Gemeinschaftsspiel \textit{"Ganz schön clever"} ist ein Würfelspiel, das eine hohe Komplexität aufweist. Diese kommt vor allem durch die hohe Anzahl an möglichen Aktionen (dem sogenannte Aktionsraum) für den Spieler und die vielen Zusammenhänge des Belohnungssystems im Spiel zustande. Das Spiel weist zusätzlich eine hohe Stochastizität auf, welche die Komplexität weiter erhöht.

Interessant ist, wie man für ein solches komplexes Spiel einen Bot oder eine Künstliche Intelligenz entwickeln kann, um dieses gut spielen zu können. Dabei könnte die Komplexität der Aufgabenstellung möglicherweise zu groß sein, um vom Bot erfasst zu werden und es können sich zahlreiche weitere Problemstellungen ergeben, für die Lösungen gefunden werden müssen.
\subsection{Zielsetzung und Motivation}
Ziel der Arbeit ist es, einen Bot beziehungsweise eine Künstliche Intelligenz zu entwickeln, welche das Spiel \textit{"Ganz schön clever"} möglichst gut spielen kann. Gut heißt hierbei, dass im Durchschnitt eine möglichst hohe Punktezahl im Spiel erzielen wird. Zum Vergleich werden hierbei Punktezahlen von menschlichen Spielern erhoben und mit der Leistung der Künstlichen Intelligenz verglichen. Es soll analysiert werden, welche Aspekte es bei der Entwicklung zu beachten gilt und wie sich unterschiedliche Ansätze auf das Verhalten und die Performanz der Künstlichen Intelligenz auswirken.

Es gibt noch keine Untersuchungen zu einer Spiel-KI für das Spiel \textit{"Ganz schön clever"}, daher ist es interessant Erkenntnisse darüber zu gewinnen, welche Schwierigkeiten sich hierbei ergeben und wie diese überwunden werden können.
\subsection{Aufgabenstellung}
Es ist eine Künstliche Intelligenz für das Spiel \textit{"Ganz schön clever"} zu implementieren. Hierbei sollen der Vorgang sowie die Ergebnisse des Prozesses analysiert und bewertet werden. Hierzu wird zunächst ein Prototyp entwickelt, welcher eines der fünf Felder des Spiels beinhaltet. Dieser Prototyp soll Einsichten über die Machbarkeit und die Rahmenbedingungen des Projektes bringen. Daraufhin werden die Künstliche Intelligenz und die Spielumgebung schrittweise um ihre jeweiligen Funktionalitäten erweitert, bis das Spiel vollständig und möglichst gut von der Künstlichen Intelligenz gespielt werden kann.
\subsection{Aufbau der Arbeit}
Kapitel 2 beinhaltet einen Grundlagenteil, in dem zunächst das Spiel \textit{"Ganz schön clever"} und seine Mechaniken erklärt werden. Anschließend werden maschinelles Lernen, Reinforcement Learning, Deep Learning und Proximal Policy Optimization behandelt, um eine fachliche Grundlage für das Verständnis der Abläufe beim Training der Künstlichen Intelligenz zu bieten. Im Folgenden werden die verwendeten Technologien des Projektes behandelt. Die vorgestellten Technologien sind Gynmansium, Stable Baselines 3, Matplotlib und ChatGPT 4.

Kapitel 3 befasst sich mit den Anforderungen und der Konzeption des Projektes. Es werden Rahmenbedingungen des Projektes sowie die Anforderungen und die Konzeption für die Spielumgebung und die Künstliche Intelligenz behandelt.

Kapitel 4 zeigt und erklärt die Implementierung des Projektes. Dabei werden die Variablen und Methoden des Projektes erläutert und beschrieben. Zu einem großen Teil wird Pseudocode verwendet, um die Implementierung verständlicher zu machen und den Umfang zu begrenzen.

Kapitel 5 befasst sich mit den Ergebnissen des Projektes, wobei zunächst der Verlauf der Implementierung und dessen einzelne Meilensteine behandelt werden. Ein Prototyp wurde erstellt und anschließend schrittweise erweitert. Daraufhin folgen die Analyse und Darstellung des finalen Modells.

Das abschließende Kapitel 6 beinhaltet eine Zusammenfassung der Arbeit und Anreize zur Weiterarbeit am Projekt.