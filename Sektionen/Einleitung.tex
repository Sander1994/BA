\section{Einleitung}
\subsection{Hinführung zum Thema}
In den vergangenen Jahren gewann Künstliche Intelligenz und insbesondere maschinelles Lernen zunehmend an Bedeutung, Tendenz steigend \cite{noauthor_kunstliche_nodate,noauthor_maschinelles_2023}. Im Jahr 2023 ist eines der präsentesten neuen Tools ChatGPT 4. Dieses Tool ist ein Chat-Bot, welcher es dem Benutzer ermöglicht mit ihm zu kommunizieren und ihm Fragen oder Aufgaben zu stellen. Solche Tools ermöglichen es Benutzern zunehmend ihre Tätigkeiten zu vereinfachen und prägen somit das Leben der Menschen zunehmend \cite{tagesschaude_chatgpt_nodate}. Auch in dieser Arbeit wurde ChatGPT 4 als unterstützendes Tool verwendet. Es wurde vor Allem dafür benutzt um fachliche Fragen zu beantworten, aber auch anfangs um Code für den Prototypen zu generieren. Mit zunehmender Komplexität der zu bearbeitenden Aufgabe sinkt die Verlässlichkeit solcher Tools. Daher ist es wichtig die Anfragen an den Chat-Bot möglichst präzise zu formulieren und den Aufgabenbereich angemessen einzuschränken um das Tool nicht zu überfordern.

Auch in der Spielentwicklung nimmt die Künstliche Intelligenz und das maschinelle Lernen schon seit langem eine zentrale Rolle ein \cite{noauthor_kunstliche_2023}. In vielen Spielen gibt es eine oder mehrere Künstliche Intelligenzen, welche bestimmte Aufgaben erfüllen, um den Spieler bei Spiel zu unterstützen oder ihm als Widersacher zu dienen. Auch hier gilt je komplexer die Aufgabenstellung desto schwieriger ist es einen solchen Bot zu generieren, welcher diese gut und richtig lösen kann.

Das Gemeinschaftsspiel "Ganz schön clever" ist ein Würfelspiel, welches eine hohe Komplexität aufweist. Diese kommt vor allem durch die hohe Anzahl an möglichen Aktionen (dem sogenannte Aktionsraum) für den Spieler und die vielen Zusammenhänge des Belohnungssystems im Spiel zu Stande. Das Spiel weist zusätzlich eine hohe Stochastizität auf, welche die Komplexität weiter erhöht.

Interessant ist wie man für ein solch komplexes Spiel einen Bot oder eine Künstliche Intelligenz entwickeln kann um dieses gut spielen zu können. Dabei könnte die Komplexität der Aufgabenstellung möglicherweise zu groß sein, um vom Bot erfasst zu werden und es können sich zahlreiche weitere Problemstellungen ergeben, für die Lösungen gefunden werden müssen.
\subsection{Zielsetzung und Motivation}
Ziel der Arbeit ist es einen Bot beziehungsweise eine Künstliche Intelligenz zu entwickeln, welche das Spiel "Ganz schön clever" möglichst gut spielen kann. Gut heißt hierbei, dass er im Durchschnitt eine möglichst hohe Punktezahl im Spiel erzielen kann. Zum Vergleich werden hierbei Punktezahlen von menschlichen Spielern erhoben und mit seiner Leistung verglichen. Es soll analysiert werden, welche Aspekte es bei der Entwicklung zu beachten gilt und wie sich unterschiedliche Ansätze auf das Verhalten und die Performance des Modells (des Bots) auswirken.

Es gibt noch keine Untersuchungen zu einer Spiel-KI für das Spiel "Ganz schön clever" daher ist es interessant Erkenntnisse darüber zu gewinnen welche Schwierigkeiten sich hierbei ergeben und wie man diese überwinden kann.
\subsection{Aufgabenstellung}
Es ist eine KI für das Spiel "Ganz schön clever" zu implementieren. Hierbei sollen der Vorgang sowie Ergebnisse des Prozesses analysiert und bewertet werden. Hierzu wird zunächst ein Prototyp entwickelt, welcher eines der fünf Felder des Spiels beinhaltet. Dieser soll Einsichten über die Machbarkeit und die Rahmenbedingungen des Projektes geben. Daraufhin werden das Modell und die Spielumgebung schrittweise um ihre jeweiligen Funktionalitäten erweitert, bis das Spiel vollständig und möglichst gut von der Künstlichen Intelligenz gespielt werden kann.
\subsection{Aufbau der Arbeit}
Die Arbeit beginnt mit einem Grundlagenteil, bei dem zunächst das Spiel Ganz schön clever und seine Mechaniken erklärt werden. Anschließend werden Maschinelles Lernen, Reinforcement Learning, Deep Learning und Proximal Policy Optimization behandelt, um eine Grundlage für das Verständnis der Abläuft beim Training der Künstlichen Intelligenz zu bieten. Im folgenden werden die verwendeten Technologien des Projektes behandelt. Die vorgestellten Technologien sind Gynmansium, Stable Baselines 3, Matplotlib und ChatGPT 4.

Kapitel 3 befasst sich mit den Anforderungen und der Konzeption des Projektes. Es werden Rahmenbedingungen des Projektes sowie die Anforderungen und die Konzeption für die Spielumgebung und die Künstliche Intelligenz besprochen.

Kapitel 4 zeigt und erklärt die Implementierung des Projektes. Dabei werden die Variablen und Methoden des Projektes erläutert und beschrieben. Zum einem großen Teil wird Pseudocode verwendet, um die Implementierung verständlicher zu machen und den Umfang zu begrenzen.

Kapitel 5 befasst sich mit den Ergebnissen des Projektes. Zunächst wird der Implementierungsverlauf behandelt. Es wurde ein Prototyp erstellt und dieser wurde dann Stück für Stück erweitert. Daraufhin folgen die Analyse und Zurschaustellung des finalen Modells.

Zum Abschluss folgt eine Zusammenfassung und Anreize zur Weiterarbeit am Projekt.