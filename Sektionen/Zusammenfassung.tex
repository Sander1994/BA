\section{Zusammenfassung}

Zunehmend prägen das maschinelle Lernen und KI die Arbeit und das Leben der Menschen. Besonders präsent sind im Jahr 2023 unter Anderem potente Chatbots wie ChatGPT 4. Solche Tools ermöglichen es Benutzern komplexe sowie komplizierte Aufgaben deutlich einfacher und schneller abzuarbeiten. Hervorzuheben ist hierbei auch, dass man mit solchen Tools deutlich weniger Fachwissen benötigt um in einem Bereich aufgaben effizient lösen zu können, da es einem eine Vielzahl von Informationen zum gewünschten Thema auf anfrage bereitstellen kann. Je komplexer das Problem oder die Fragestellung allerdings sind, desto unverlässlicher werden diese Tools. Man muss seine Anfragen deshalb möglichst präzise formulieren und die Problemstellung in für das Tool angemessene Teilaufgaben zerlegen.

Auch in der Spielentwicklung spielen maschinelles Lernen und KI schon seit langem eine bedeutende Rolle. In den meisten Spielen gibt es sogenannte Bots, welche man als KI bezeichnen kann. Diese sollen bestimmte Aufgaben im Spiel erfüllen um den Spieler zu unterstützen oder im als Widersacher zu dienen. Je komplexer die Aufgabe, umso schwerer ist es einen solchen Bot zu erstellen, welcher die Aufgabe auf zufriedenstellende Weise erfüllen kann.

Das Gesellschaftsspiel "Ganz schön clever" ist ein Würfelspiel, welches eine hohe Komplexität aufweist. Diese kommt vor allem durch die vielen Aktionsmöglichkeiten des Spielers und die multiplen zusammenhängen innerhalb des Belohnungssystems zustande. Außerdem weist es eine hohe Stochastizität auf, welche die Komplexität weiter erhöht.
Ziel dieser Arbeit ist es eine KI beziehungsweise einen Bot für dieses Spiel zu entwickeln, der das Spiel effizient spielen kann, sowie zu analysieren welche Aspekte der Entwicklung dabei relevant und zu beachten sind.

Dazu mussten Spielumgebung sowie KI zunächst implementiert werden. Dies geschah mithilfe von Bibliotheken wie Stable-Baselines3 und Gymnasium.
Insgesamt ergab sich dabei, dass sich mithilfe des PPO-Algorithmus von Stable-Balseslines3 auf relativ einfache Weise ein effizientes Modell für das Spiel entwickeln lässt.